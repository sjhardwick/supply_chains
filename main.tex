
\documentclass{article}
\usepackage[utf8]{inputenc}
\usepackage{titling}

\PassOptionsToPackage{hyphens}{url}\usepackage{hyperref} % URLs
\usepackage{amsmath} % For maths
\usepackage{palatino} % Font
\usepackage[margin = 3.81cm]{geometry} % Set margins
\usepackage{parskip} % Space between paragraphs
\usepackage{amsfonts}

% Packages for tables
\usepackage{makecell} % For tables
\usepackage{multirow} % For merging cells vertically in tables
\usepackage{rotating} % For landscape tables
\usepackage{tabularx}
\usepackage{longtable}
\usepackage[flushleft]{threeparttable} % For table notes
\usepackage{threeparttablex} % For table notes in long tables
\usepackage{booktabs}
\usepackage{afterpage}

% Settings for bibliography
\usepackage[authordate, backend = biber, isbn = false, maxcitenames = 2, style = apa]{biblatex}
\addbibresource{scr-bib.bib} 

\title{International Implications of Industrial Policies for Supply Chain Resilience}
\author{Sam Hardwick}
\date{Draft Last Updated: 23 August 2024}

\begin{document}

\begin{titlingpage}
    \maketitle
    \begin{abstract}
        Governments are increasingly pursuing industrial policies with supply chain resilience as their stated objective. These policies can have both positive and negative welfare implications, domestically and internationally. We outline key cases for and against using industrial policies for supply chain resilience and provide an overview of current issues regarding the international governance of these policies. We then present a simple three-country model with sequential production to shed light on typical policies and their international spillovers. We conclude by discussing some proposals for cooperation on governing supply chain resilience policies internationally.
    \end{abstract}
\end{titlingpage}

\section{Introduction}

Industrial policies in major economies have proliferated in size and number. The trend has been particularly pronounced for high-income countries since 2020 \parencite{ilyina_industrial_2024}. The industrial policy upsurge has coincided with and responded to a growing policy interest in supply chain resilience in the wake of the COVID-19 pandemic and amid deepening great power rivalry.

COVID-19 showed that global value chains (GVCs) are susceptible to disruption and that shocks to downstream or upstream sectors can reverberate throughout the chain, sometimes with macroeconomic consequences. As pandemic risks abated, geopolitical and security-driven risks to supply chains became the object of focus. At the same time, risks related to climate change and extreme weather events have continued to rise.

To what extent are industrial policies for supply chain resilience justified? Economists have tended to approach industrial policy with a dose of scepticism. At the same time, the field has always been concerned with identifying and recognising market failures that present a role for public policy. The literature and public discussion about supply chain resilience exemplifies both.

While shocks are transmitted through GVCs, as a key macroeconomic study of pandemic lockdowns points out, 'renationalising' these chains does not, in general, make economies more resilient to those shocks \parencite{bonadio_global_2021}. Access to international trade is a buffer against domestic shocks and access to multilateral trade is a buffer against bilateral shocks. (Also mention Caselli et al here.) The astonishing rebound of global merchandise trade in 2022 suggests that GVCs were probably more resilient than they may have appeared.

Research since the pandemic has also pointed to conditions in supply networks which may give rise to market failures or which make shocks more likely to propagate. These include externalities that cause insufficient firm entry, bottleneck firms that systematically underproduce inputs, and regulatory or informational issues that discourage investment in risk management (cite Elliott and Golub review, your review, key papers in footnotes). With some notable exceptions, like Grossman JPE (cite) or the PC's report, this research has tended not to address questions of optimal public policy, beyond sometimes pointing to a general role for it.

As this literature has grown, so has the footprint of industrial policies on the global economy. In particular, it is evident that subsidies, first-best or otherwise, will play a central role in driving the green energy transition, a phenomenon which is reorganising GVCs. These developments have intensified interest in supply chain resilience and industrial policy as governments seek both energy security and market power in the future low-carbon global economy. This economy demands inputs that today are concentrated in few hands: not just the critical minerals used in electric vehicles (EVs) and green energy generation, but specialised inputs to those green technologies.

The desire for diversification has been a stated motive for recent industrial policies in some countries, sometimes based on national security concerns. But diversification is not necessarily resilience, particularly if it means sacrificing optimal technology, if it comes at the price of other productive activities, or if it is pursued in a way that raises trade uncertainty or sparks retaliation. Similarly, the impacts of industrial policies on supply chain resilience are not clear cut. There are negative and positive spillovers to consider, domestically and globally. A central question, posed by \textcite{bown_modern_2024}, is whether optimal supply chain resilience policy from a domestic standpoint differs to that from a global standpoint.

To contribute to answering that question, and to better understand these policies and spillovers, we proceed as follows. The next section reviews the theoretical grounds for policies to manage supply chain risk and the ways in which these policies are governed internationally, particularly in the multilateral trading system. We then present a simple three-country model with sequential production to illustrate potential international spillovers of industrial policies, both on conventional measures of economic welfare as well as on the variance of final supply of a product, since the latter may be important from a risk tolerance perspective. We simulate four policy examples: entry subsidies for input producers, specific production subsidies, tariffs on inputs, and export bans on inputs. 

The policy examples reflect and extend earlier findings on similar models, including the existence of a small optimal entry subsidy. Subsidies that target entry costs alone preserve the international composition of production while encouraging new firms, thus reducing disruption risk, though at a cost to the subsidiser. Small production subsidies can increase global welfare and supply, especially if they are pursued by both producing countries, though they do not make final supply less volatile. Tariffs exhibit a familiar prisoner's dilemma structure, pointing to a non-cooperative equilibrium that lowers welfare and supply without offering reduced variation. Export bans can take on a zero-sum form with globally inefficient outcomes without cooperation.

In light of these theoretical results, we conclude by discussing some active policy proposals for international cooperation on supply chain resilience, and where these efforts may be best directed. Finally, we suggest some potential directions for future research.

Still to cite: Bagwell and Lee, 2018; Bown review paper (recent one).

\section{Background: Industry policy, supply chain resilience and the multilateral trading system}

\subsection{For and against industrial policy for supply chain resilience}

There are several well-known economic arguments for industrial policy at a domestic level. (Following Harrison and Rodríguez-Clare (2010) and Bown (2024), we will think of industrial policy in broad terms as deviation from policy neutrality.) Sometimes these arguments are based on internal economies of scale, such as learning by doing that occurs within a firm. More commonly, arguments refer to externalities. Benefits from technological development, knowledge generation or human capital advancement in one firm can spill over to others. In recent decades, more emphasis has been placed on industrial policy's role in discovering the costs and benefits of new economic activities, and in addressing coordination issues through public--private cooperation (Rodrik 2004).

In the subdomain of supply chain resilience, there further economic arguments for industrial policy, defined in this broad sense. All organisations and GVCs face some level of disruption risk, which can be thought of as the likelihood of some event that directly affects its usual functioning---its ability to deliver `the right products and services on time, with the required specifications, at the right place and to the right customer' (cite Carvalho et al 2012). Following Baldwin and Freeman (2022 cite), we can think of socially excessive risk as a situation where firms' decisions lead to a different bundle of risk and reward---reward such as cost savings---than public interest would demand. Socially excessive supply chain risk can be a symptom of traditional market imperfections. For example, industries with few new entrants are more likely to see bottleneck firms emerge (cite Carvalho et al. n.d.), which wield market power, restrict supply and concentrate risk.

Another key example of potentially excessive supply chain risk is modelled by \textcite{bimpikis_supply_2019}. The authors show that, under certain conditions, the equilibrium number of entrants in a supply network may be lower than the number that maximises consumer surplus and profit. In an environment where firms face random disruptions, new firms, through the act of entry, benefit the rest of their supply network by diversifying supply or demand. Firms might be able to reap some of these external benefits---by profiting off their reputation as reliable suppliers, for example---but not all of them, absent some implausibly sophisticated contracting across every firm in a network.\footnote{Other potential sources of socially excessive risk are reviewed by Elliott and Golub (2022) and by Hardwick et al. (2024).} 

Conversely, there are areas where exposure to supply chain risk may appear high, but there is not a clear policy rationale for addressing it directly. In some cases, there may be a trade-off between a lengthy, high value-adding GVC and a shorter, more expensive, less volatile one (cite Levine 2012). Some network structures, the patterns of interconnection between firms, have been shown to propagate shocks more intensely than others (cite Todo and Inoue 2019). Yet there is no case in principle for intervention unless that intervention might be expected to bring about a socially superior balance of risk.

On that basis, as the Australian Productivity Commission (cite 2021?) has outlined, policies to improve supply chain resilience are best targeted at vulnerable and essential products, where public risk appetites are lowest. `Essential' may be an ambiguous concept, but one that governments regularly deal with; for example, in strategising for disaster management or maintaining pharmaceutical or energy stockpiles. As vulnerability (or exposure to risk) and essentiality decline, private risk management is more likely to suffice. 

The role of entry barriers in generating supply chain risk suggests that industrial policy is more likely to aid resilience if it permits or encourages competition and new entrants. A corollary is that policies are more likely to reduce resilience if they limit competition, either by restricting entry or favouring non-competitive incumbents. Evidence from China indicates that subsidies to more (less) competitive sectors are more likely to boost (reduce) productivity (Aghion et al. 2015). In Japan, sectors that benefited from industrial policy (in the 1970s and 80s, confirm) tended to succeed if there was significant competition (Porter and Sakakibara 2004).

Concentration of production by country is a flawed but informative proxy for the competition issues that can lead to vulnerability. Concentrated production does not inherently mean there are barriers to entering and producing. It may be a temporary but desirable result of a firm or set of firms innovating. In some cases, however, concentration is a signal of vulnerability to shocks (e.g. some PPE or diesel exhaust fluid from China). In these cases, alternative suppliers cannot enter or scale up quickly enough to deliver the socially optimal response. Reducing risk could take the form of diversification, including through subsidies or agreements with third countries, or other risk management strategies, like emergency stockpiling and data sharing.

Choice of instrument matters and depends on institutional context. Subsidies and trade levers, though still common, are not the only tools available. Others (e.g. infrastructure, manufacturing extension, specialised training) may better serve competition and productivity goals (Juhász et al. 2023). Some instruments and levers are both obtuse and involve greater direct welfare loss to the global community (e.g. export restrictions). Trade agreements, commercial diplomacy and regulatory reform may all be beneficial for supply chain resilience to the extent that they address barriers to alternative sources of supply and demand. Sometimes more targeted, security-oriented policies like stockpiling may be reasonable alternatives, if storage and other costs are low enough.

Industrial policies have positive and negative international spillovers for supply chain resilience. On the positive side, industrial policies may accelerate the development of technology which, with sufficient factor mobility, spreads internationally. They may facilitate new exporters, diversifying the supply or customer base. On the negative side, they may erode international competition, including by limiting export-competing countries’ access to third markets. These effects lower resilience by limiting the ability of new supply to enter. They may lead to wasteful subsidies races: if one country subsidises, its competitor may be forced to subsidise too or exit markets.

Beyond subsidies, other policy instruments may have more direct implications for supply chain resilience. Local content requirements may impede responses to shocks by removing alternative supply options, or at least by delaying the ability to adjust. Export bans similarly remove an input source from global markets. Cooperation is needed to identify and measure these spillovers to inform multilateral subsidy rules, especially where trade and climate change imperatives interact (Bown 2023).

\subsection{Supply chain resilience and the world trading system}

This section briefly reviews the governance of industrial policy under the GATT and WTO and its implications for supply chain resilience. Regional and bilateral supply chain initiatives, which have proliferated but remain nascent in their activity. Ways to share information and plan crisis response. Urea debacle. When it comes to the WTO, views differ on how subsidies should be treated, but there is broad recognition of a need for better data, a better understanding of spillovers and a shared understanding of principles for subsidies governance.

Origin of the the MTS system itself is resilience policy. It is security--economic.

Driven partly by the need for green technology, there is growing recognition that trade rules should recognise certain subsidies as acceptable, first-best policies.

There is a range of views on how the WTO should approach industrial policy. At one end, Sykes (2003, 2010) argues the best approach to subsidies in the WTO may be retaining non-violation rules but otherwise going laissez-faire.

Bagwell and Staiger.

Hoekman and Nelson (2020) highlight the need for an international work program on to identify common principles (e.g. citizen welfare) and develop simple, robust rules of thumb for subsidies governance.

Bown and Hillman (2019), writing about China’s industrial policy, argue that work on measurement and identification of subsidies is critical. Beyond that, a ‘green/amber/red light’ system, with an expanded set of prohibited subsidies, would be ideal. They also raise an alternative approach of introducing competition policy concepts to the WTO (e.g. notifications on size and competition among large firms).

Bown and Clausing (2023), writing about US, European and Chinese climate policies, recommend broadly accepting subsidies while implementing some guardrails (e.g. prohibiting export restrictions and content requirements; minimising subsidies with competition spillovers). Raise Indonesia's downstreaming policies in this context.

Much discussion focuses on China. Shipbuilding example, which is a major effort towards measuring international impact of subsidies. Solar PV cell example. IEA. But compare with Danish support for wind tech. Broadly beneficial and transmitted through FDI. Key point: markets stayed open, including to investment. Multilateral trade held.

China has used industrial policy to develop solar PV manufacturing from the mid-2000s. By 2015, it had become the world’s top producer. Support has taken the form of tax breaks, preferential lending, subsidised land from local governments, and cash grants from municipal and provincial governments (Houde and Wang 2022). There have been large, clear positive impacts on emissions reduction and innovation (Xu et al. 2022). At the same time, overcapacity issues, coupled with high concentration at firm and facility levels, have led to concerns about global supply chain risk (IEA 2022; Wang et al. 2014).  

Denmark began producing wind turbines in the 1980s, supported by price guarantees and favourable tax treatment, though not by trade restrictions. Investment subsidies for turbines were removed during the 1980s, production subsidies in 2000 (van Est 2022). Studies suggest the subsidies enabled learning-by-doing effects and with benefits likely exceeding their future discounted costs (Hansen et al. 2003).   

Danish technology and knowhow spread internationally, notably through FDI in Spain and Germany. In 2002, 92 per cent of exported wind turbines (by value) came from Denmark. By 2012, Germany accounted for 38 per cent, Denmark 22 per cent and Spain 13 per cent. Some of this reduction in Danish market share stemmed from overcapacity issues around the global financial crisis. Despite initially high concentration, concerns about competition and supply chain risk were muted (e.g. no US trade action, no IEA commentary). 

The need for better data and modelling on spillovers (on supply chain resilience and the much bigger problem of emissions reduction) is widely recognised as a priority. Next section presents a simple model to illustrate some of these spillovers.

\section{A model of trade with input disruptions}

The model below serves to illustrate potential spillovers from industrial policies for supply chain resilience by providing some simple stylised examples. The aim is not to present sweeping or definitive policy conclusions, but to highlight some potential mechanisms through which one country's policies might help or harm others' resilience, and areas where international cooperation may lead to collective benefit. It also provides a novel approach to integrating stochastic supply chain risks into simple trade policy models.

The model has two sectors, an input $I$ and a final good $F$, and three countries, $A$, $B$ and $C$. Two-country models can understate the negative international spillovers from subsidies (Bown and Hillman, check). One country's subsidies can carve out a competitor's market share in a third country to which they both export. Here, $C$ does not produce either good, but represents the third market where $A$ and $B$ compete, thereby capturing some of these spillovers. It proceeds in two stages. 

First, firms decide whether or not to enter the market for inputs. In doing so, they consider the prices they will face, which depend on policies, their risk of a disruption, the costs of trade and the price of raw materials. They also consider consumers' demand for the final product, which depends on incomes and the elasticity of substitution between domestic and foreign goods. To enter the market, they need to pay an entry cost $\kappa$ and purchase raw materials on a global exchange at price $p_r$, taken as given. If their expected profits are positive, they will enter.

Once they have entered, whether or not they produce and sell inputs depends on a stochastic disruption risk, $d_j \in (0, 1)$. A firm in country $j$ will make inputs with probability $1 - d_j$, but if it is disrupted, it will not produce and the raw materials it purchased will not be used. This process echoes the model of \textcite{bimpikis_supply_2019}.

After the entry process, prices are considered set, and production occurs. The input producers purchase raw materials and either make inputs or experience disruption. These inputs are sold to firms in the final good sector in countries $A$ and $B$, transformed into final goods, then sold to customers in countries $A$, $B$ and $C$. Welfare is evaluated based on consumption of the final good and income after any revenue from tariffs or costs of subsidies have been accounted for.

Cite Melitz and Redding (2014). Model of sequential production with a similar setup that shows how trade creates a reorganisation of production that boosts domestic productivity, with productivity increasing with the number of stages. The setup is also similar to that of \textcite{bagwell_chapter_2016} and \textcite{venables_trade_1987}.

\subsection{Consumers}
 
Each country $j$ has a representative consumer with constant elasticity of substitution (CES) preferences. These consumers maximise utility given by
\begin{equation}
    U_j = \left[ (x_{FAj})^{\frac{\sigma - 1}{\sigma}} + (x_{FBj})^{\frac{\sigma - 1}{\sigma}} \right]^{\frac{\sigma}{\sigma - 1}}
\end{equation}
where $x_{Fk}^j$ represents consumption in country $j$ of final goods produced by country $k$. The elasticity of substitution between goods from different countries is $\sigma > 1$. Consumers face the budget constraint 
\begin{equation}
    p_{FA} \bar{\tau}^{j}_{FA} x_{FAj} + p_{FB} \bar{\tau}^{j}_{FB} x_{FBj} \leq M_j
\end{equation}
where $p_{Fi}$ is the price of final goods produced in country $i$ and $M_j$ is income in country $j$. $\bar{\tau}_{Fij}$ captures trade costs faced by importers from $j$ of final goods from $i$. This term takes the value 
\begin{equation} \label{eq:tau}
    \bar{\tau}_{Fij} =
    \begin{cases}
        1 &\text{if } i = j \\
        (1 + t_{Fij})(1 + \tau) &\text{if } i \neq j
    \end{cases}
\end{equation}
where $t^j_{F}$ is the import tariff rate in country $j$, if one applies, and $\tau$ is a generic international trade cost, like freight. 

As shown by \textcite{dixit_monopolistic_1977}, a price index for final goods consumed in country $j$ can be defined as
\begin{equation} \label{eq:pbar_f}
      \bar{P}_{Fj} = \left[ (p_{FA} \bar{\tau}_{FAj} )^{1-\sigma} +  (p_{FB} \bar{\tau}_{FBj})^{1-\sigma} \right]^\frac{1}{1-\sigma} .
\end{equation}
Using the price index in (\ref{eq:pbar_f}), demand for final goods produced in country $k$ and consumed in country $j$ can be written as
\begin{equation}
    x_{Fk}^{j} = M_j \left( \frac{p_{Fk}^{j}}{\bar{P}_F^{j}} \right)^{-\sigma}  .
\end{equation}
For simplicity, we assume the elasticity is constant across all stages of production, though it could be allowed to vary. Our baseline considers an elasticity $\sigma$ equal to 5.

Welfare is evaluated, among other measures, using the indirect utility function
\begin{equation}
    V_j = I_j \bar{P}_{Fj}
\end{equation}
where $I_j$ equals $M_j$ plus any net revenue from tariffs, costs from subsidies, and any profits in the input sector in country $j$.

\subsection{Final good sector}

The final good sector consists of a single representative firm per country, one in country $A$ and one in country $B$. Country $C$ consumes final goods but does not produce them. Since we are interested in market imperfections upstream---supply chain disruptions---we assume perfect competition in the final good sector. Each firm in the sector sources a mix of domestic and foreign inputs and produces according to CES technology, given by
\begin{equation}
    Q_{Fj} = \left[ \left( x_{IAj} \right)^\frac{\sigma-1}{\sigma} + \left( x_{IBj} \right)^\frac{\sigma-1}{\sigma} \right]^\frac{\sigma}{\sigma-1}
\end{equation}
where $x^j_{Iij}$ are inputs sourced by country $j$ from country $i$ for production of goods $Q_{Fj}$. 

They sell these goods to consumers in the three countries. Their costs are
\begin{equation}
    C(Q_{Fj}) = p_{IA} \bar{\tau}_{IAj} x_{IAj} + p_{IB} \bar{\tau}_{IBj} x_{IBj}
\end{equation}
where input trade costs $\bar{\tau}$ are defined analogously with the final good trade costs in (\ref{eq:tau}). 

With perfect competition and constant returns to scale, price is equal to unit cost
\begin{equation}
    p_{Fj} = \left[ ( p_{IA} \bar{\tau}_{IAj} )^{1 - \sigma} + ( p_{IB} \bar{\tau}_{IBj} )^{1 - \sigma}) \right]^{\frac{1}{1 - \sigma}}
\end{equation}
with conditional demand for inputs then given by
\begin{equation} \label{eq:input_demand}
    x_{Iij} = Q_{Fj} \left( \frac{p_{Fj}}{p_{Ii}} \right)^{\sigma}.
\end{equation}

\subsection{Input sector}

Firms in the input sector procure raw materials, which are traded freely on a global commodity market at the price $p_r$. They transform these materials directly into inputs unless they experience a disruption. The probability of a disruption for firms in country $i$ is the parameter $d_i \in (0, 1)$. Other than disruption risk, which varies by country, input-producing firms face identical technologies and costs.

Each firm's expected production can be written as
\begin{equation}
    \mathbb{E} (q_{Ii}) = (1 - d_i) r_i
\end{equation}
where $r_i$ is the quantity of raw materials purchased by each firm. Prospective input producers decide whether to enter the market based on expected profits, which are given by
\begin{equation} \label{eq:input_profit}
    \mathbb{E} (\pi_{Ii}) = \max_{p_{Ii}} \left\{ (p_{Ii} + s_{Ii}) (1 - d_i) r_{i} - p_r r_i - (1 - e_i) \kappa \right\}
\end{equation}
where $s_{Ii}$ is an optional specific production subsidy, applied per unit of input produced by firms in country $i$. $e_i$ is the rate of an optional entry subsidy and $\kappa$ is the fixed entry cost parameter. Once firms purchase raw materials, they may be used for production, but if there is a disruption, production cannot occur and the materials are sunk costs.

Noting that the total quantity of inputs produced by country $i$ equals the sum of $x_{IiA}$ and $x_{IiB}$, we can substitute the conditional demands in (\ref{eq:input_demand}) into the profit function in (\ref{eq:input_profit}). Doing so and solving for the profit-maximising price yields
\begin{equation}
    p_{Ii} = \frac{\sigma}{\sigma - 1} \left( \frac{p_r}{1 - d_j} - s_{Ii} \right) .
\end{equation}
Based on this price, firms in each country enter sequentially until the expected profits from one additional firm turn negative, given the fixed entry cost. If expected profits are still negative with only one firm, that country will not produce any inputs. If firms in neither country expect a profit, no raw materials are purchased and nothing is produced.

Because there is free entry up until raw materials are purchased, profits in the input sector are essentially negligible. Yet they may be slightly above zero because the number of firms is required to be an integer, so that the risk of disruption can be simulated for each firm. Accordingly, any profits are added to the value function above (ref) when computing welfare.

\subsection{Simulation of policies}

Once firms have procured raw materials, disruptions either occur or do not occur to each producer, and production begins. This section analyses four simple policy examples through simulations of this process. Since we are interested not just in expected welfare, but in the variation of outcomes, for each example we typically present:
\begin{itemize}
    \item expected welfare, through indirect utility $\mathbb{E} (V_j)$, 
    \item expected production of inputs, $\mathbb{E} (Q_{Ii})$ and final goods, $\mathbb{E} (Q_{Fi})$, 
    \item expected consumption of final goods, $\mathbb{E} (X_{Fj})$, 
    \item the coefficient of variation (standard error of the mean divided by the mean) for the consumption of final goods, $CV (X_{Fj})$, and
    \item the probability of a shortfall, which we define as a drop in consumption to less than half of the level that would be expected in the absence of any policy intervention.
\end{itemize}
In other words, $\mathbb{P}(\text{Shortfall}) = \mathbb{P} (X / \mathbb{E}(X_0) < 0.5)$, where $X$ is actual consumption under the policy scenario of interest, and $\mathbb{E} (X_0)$ is expected consumption with no subsidies or trade restrictions.

\begin{table}
    \centering
    \begin{threeparttable}
        \renewcommand{\arraystretch}{1.3}
        \caption{Entry subsidy example}
        \label{tab:entry_subsidy}
        \vspace{1mm} 
        \begin{tabular}{lrrrrr}
            \toprule
            & \multicolumn{5}{c}{Entry subsidy rates} \\
            & \makecell[c]{None} & \multicolumn{2}{c}{Unilateral} & \multicolumn{2}{c}{Symmetric} \\
            \cmidrule{2-2} \cmidrule{3-4} \cmidrule{5-6}
            & $e_A = 0$ & $e_A = 0.2$ & $e_A = 0.4$ & $e_A = 0.2$ & $e_A = 0.4$ \\
            & $e_B = 0$ & $e_B = 0$ & $e_B = 0$ & $e_B = 0.2$ & $e_B = 0.4$\\
            \midrule
            \textbf{Country A} \\
            Input producers & 2 & 3 & 4 & 3 & 4 \\ 
            $\mathbb{E}(V_A)$ & 20.05 & 18.18 & 16.31 & 18.18 & 16.31 \\
            $CV(X_A)$ & 17.47\% & 15.73\% & 14.81\% & 13.91\% & 11.98\% \\
            $\mathbb{P}(\text{Shortfall})$ & 2.00\% & 1.59\% & 1.06\% & 0.26\% & 0.05\% \\ 
            \midrule
            \textbf{Country B} \\
            Input producers & 2 & 2 & 2 & 3 & 4 \\ 
            $\mathbb{E}(V_B)$ & 20.05 & 20.05 & 20.05 & 18.18 & 16.31 \\
            $CV(X_B)$ & 17.45\% & 15.96\% & 15.17\% & 13.91\% & 11.98\% \\
            $\mathbb{P}(\text{Shortfall})$ & 2.00\% & 1.59\% & 1.06\% & 0.26\% & 0.05\% \\ 
            \midrule
            \textbf{Country C} \\
            $\mathbb{E}(V_C)$ & 17.73 & 17.73 & 17.73 & 17.73 & 17.73 \\
            $CV(X_C)$ & 17.46\% & 15.83\% & 14.98\% & 13.90\% & 11.98\% \\
            $\mathbb{P}(\text{Shortfall})$ & 2.00\% & 1.59\% & 1.06\% & 0.26\% & 0.05\% \\ 
            \midrule
            \textbf{All countries} \\
            Input producers & 4 & 5 & 6 & 6 & 8 \\
            $\mathbb{E}(V)$ & 57.82 & 55.96 & 54.09 & 54.09 & 50.35 \\
            $CV(X)$ & 17.46\% & 15.83\% & 14.98\% & 13.90\% & 11.98\% \\
            $\mathbb{P}(\text{Shortfall})$ & 2.00\% & 1.59\% & 1.06\% & 0.26\% & 0.05\% \\ 
            \bottomrule
        \end{tabular}
        \begin{tablenotes}
            \small \item Note: Calculated with $p_R = 0.5$, $\sigma = 5$, $\tau = 0.1$, $\kappa = 1$ and $d_A = d_B = 0.1$. Income for all countries is set to 10. Variation and probability are based on 30 samples of 20,000 simulations each. Bootstrap standard errors all $<0.01\%$.
        \end{tablenotes}
    \end{threeparttable}
\end{table}

Note that with entry subsidies, expected consumption and expected production are unchanged, since marginal revenues and costs are the same. The range, however, does change as more firms enter and spread risk. The small symmetric subsidy provides a greater reduction in variation than the large unilateral one, and at lower total subsidy cost ($0.2 \times 8 < 0.4 \times 5$). Global expected welfare and the number of entrants are the same. 

\begin{table}
    \centering
    \begin{threeparttable}
        \renewcommand{\arraystretch}{1.3}
        \caption{Specific input production subsidy example}
        \label{tab:input_subsidy}
        \vspace{1mm} 
        \begin{tabular}{lrrrrr}
            \toprule
            & \multicolumn{5}{c}{Subsidy per unit of input} \\
            & \makecell[c]{None} & \multicolumn{2}{c}{Unilateral} & \multicolumn{2}{c}{Symmetric} \\
            \cmidrule{2-2} \cmidrule{3-4} \cmidrule{5-6}
            & $s_A = 0$ & $s_A = 0.05$ & $s_A = 0.1$ & $s_A = 0.05$ & $s_A = 0.1$ \\
            & $s_B = 0$ & $s_B = 0$ & $s_B = 0$ & $s_B = 0.05$ & $s_B = 0.1$\\
            \midrule
            \textbf{Country A} \\
            Input producers & 2 & 3 & 3 & 2 & 2 \\ 
            $\mathbb{E}(V_A)$ & 20.05 & 17.65 & 15.74 & 19.81 & 18.98 \\
            $\mathbb{E}(Q_{IA})$ & 19.66 & 25.63 & 33.01 & 21.61 & 23.98 \\
            $\mathbb{E}(Q_{FA})$ & 23.25 & 25.54 & 28.46 & 25.55 & 28.36 \\
            $\mathbb{E}(X_{FA})$ & 15.80 & 16.70 & 17.93 & 17.36 & 19.26 \\
            $CV(X_{FA})$ & 17.51\% & 15.50\% & 17.54\% & 17.44\% & 17.44\% \\
            $\mathbb{P}(\text{Shortfall})$ & 1.99\% & 0.85\% & 0.39\% & 1.14\% & 0.36\% \\ 
            \midrule
            \textbf{Country B} \\
            Input producers & 2 & 2 & 1 & 2 & 2 \\ 
            $\mathbb{E}(V_B)$ & 20.05 & 20.07 & 22.43 & 19.81 & 18.98 \\
            $\mathbb{E}(Q_{IB})$ & 19.66 & 16.00 & 12.25 & 21.61 & 23.98 \\
            $\mathbb{E}(Q_{FB})$ & 23.25 & 23.54 & 23.92 & 25.55 & 28.36 \\
            $\mathbb{E}(X_{FB})$ & 15.80 & 16.57 & 17.66 & 17.36 & 19.26 \\
            $CV(X_{FB})$ & 17.51\% & 15.42\% & 17.56\% & 17.45\% & 17.44\% \\
            $\mathbb{P}(\text{Shortfall})$ & 1.99\% & 0.85\% & 0.39\% & 1.17\% & 0.36\% \\ 
            \midrule
            \textbf{Country C} \\
            $\mathbb{E}(V_C)$ & 17.73 & 18.67 & 19.96 & 19.49 & 21.62 \\
            $\mathbb{E}(X_{FC})$ & 14.91 & 15.70 & 16.80 & 16.39 & 18.18 \\
            $CV(X_{FC})$ & 17.50\% & 15.46\% & 17.54\% & 17.44\% & 17.43\% \\
            $\mathbb{P}(\text{Shortfall})$ & 1.99\% & 0.85\% & 0.39\% & 0.36\% & 0.36\% \\ 
            \midrule
            \textbf{All countries} \\
            Input producers & 4 & 5 & 4 & 4 & 4 \\
            $\mathbb{E}(V)$ & 57.82 & 56.39 & 58.13 & 59.11 & 59.59 \\
            $\mathbb{E}(Q_{I})$ & 39.32 & 41.63 & 45.26 & 43.21 & 47.95 \\
            $\mathbb{E}(Q_{F})$ & 46.50 & 48.97 & 52.39 & 51.10 & 56.71 \\
            $CV(Q_{F})$ & 17.50\% & 15.46\% & 17.54\% & 17.44\% & 17.43\% \\
            $\mathbb{P}(\text{Shortfall})$ & 1.99\% & 0.85\% & 0.39\% & 0.36\% & 0.36\% \\ 
            \bottomrule
        \end{tabular}
        \begin{tablenotes}
            \small \item Note: Calculated with $p_R = 0.5$, $\sigma = 5$, $\tau = 0.1$, $\kappa = 1$ and $d_A = d_B = 0.1$. Income for all countries is set to 10. Variation and probability are based on 30 samples of 20,000 simulations each. Bootstrap standard errors all $<0.01\%$.
        \end{tablenotes}
    \end{threeparttable}
\end{table}

Briefly note that there is a very small symmetric entry subsidy that raises global welfare, along the lines of Bagwell and Lee (2018), but there is no impact on resilience in this model since it is not large enough to induce entry.

None of the input production subsidies were very good for resilience. They do tend to enhance global welfare, noting that in this simple example, they are not drawing resources from elsewhere in the economy. Under unilateral subsidies, the variation of production tended to increase compared to the no-subsidy case.

\begin{table}
    \centering
    \begin{threeparttable}
        \renewcommand{\arraystretch}{1.3}
        \caption{Input tariff example}
        \label{tab:input_tariff}
        \vspace{1mm} 
        \begin{tabular}{lrrrrr}
            \toprule
            & \multicolumn{5}{c}{Ad valorem tariff rate on inputs} \\
            & \makecell[c]{None} & \multicolumn{2}{c}{Unilateral} & \multicolumn{2}{c}{Symmetric} \\
            \cmidrule{2-2} \cmidrule{3-4} \cmidrule{5-6}
            & $t_A = 0$ & $t_A = 0.1$ & $t_A = 0.2$ & $t_A = 0.1$ & $t_A = 0.2$ \\
            & $t_B = 0$ & $t_B = 0$ & $t_B = 0$ & $t_B = 0.1$ & $t_B = 0.2$\\
            \midrule
            \textbf{Country A} \\
            Input producers & 2 & 2 & 3 & 2 & 2 \\ 
            $\mathbb{E}(V_A)$ & 20.05 & 20.64 & 19.02 & 19.95 & 19.72 \\
            $\mathbb{E}(Q_{IA})$ & 19.66 & 21.03 & 21.99 & 19.29 & 19.19 \\
            $\mathbb{E}(Q_{FA})$ & 23.25 & 20.97 & 19.42 & 22.46 & 21.92 \\
            $\mathbb{E}(X_{FA})$ & 15.80 & 15.44 & 15.21 & 15.26 & 14.89 \\
            $CV(X_{FA})$ & 17.50\% & 17.73\% & 15.40\% & 17.51\% & 17.43\% \\
            $\mathbb{P}(\text{Shortfall})$ & 2.00\% & 1.18\% & 0.84\% & 2.02\% & 1.19\% \\ 
            \midrule
            \textbf{Country B} \\
            Input producers & 2 & 2 & 2 & 2 & 2 \\ 
            $\mathbb{E}(V_B)$ & 20.05 & 19.34 & 18.91 & 19.95 & 19.72 \\
            $\mathbb{E}(Q_{IB})$ & 19.66 & 17.95 & 16.91 & 19.29 & 19.19 \\
            $\mathbb{E}(Q_{FB})$ & 23.25 & 24.80 & 25.90 & 22.46 & 21.92 \\
            $\mathbb{E}(X_{FB})$ & 15.80 & 15.66 & 15.58 & 15.26 & 14.89 \\
            $CV(X_{FB})$ & 17.50\% & 17.52\% & 15.40\% & 17.51\% & 17.43\% \\
            $\mathbb{P}(\text{Shortfall})$ & 2.00\% & 1.18\% & 0.84\% & 2.02\% & 1.19\% \\ 
            \midrule
            \textbf{Country C} \\
            $\mathbb{E}(V_C)$ & 17.73 & 17.44 & 17.25 & 17.13 & 16.72 \\
            $\mathbb{E}(X_{FC})$ & 14.91 & 14.68 & 14.53 & 14.41 & 14.06 \\
            $CV(X_{FC})$ & 17.49\% & 17.61\% & 15.36\% & 17.46\% & 17.35\% \\
            $\mathbb{P}(\text{Shortfall})$ & 2.00\% & 1.18\% & 0.84\% & 2.02\% & 2.00\% \\ 
            \midrule
            \textbf{All countries} \\
            Input producers & 4 & 4 & 5 & 4 & 4 \\
            $\mathbb{E}(V)$ & 57.82 & 57.42 & 55.17 & 57.03 & 56.16 \\
            $\mathbb{E}(Q_I)$ & 39.32 & 38.97 & 38.90 & 38.57 & 38.38 \\
            $\mathbb{E}(Q_F)$ & 46.50 & 45.77 & 45.32 & 44.93 & 43.84 \\
            $CV(Q_F)$ & 17.49\% & 17.61\% & 15.36\% & 17.46\% & 17.35\% \\
            $\mathbb{P}(\text{Shortfall})$ & 2.00\% &1.18\% & 0.84\% & 2.02\% & 2.00\% \\ 
            \bottomrule
        \end{tabular}
        \begin{tablenotes}
            \small \item Note: Calculated with $p_R = 0.5$, $\sigma = 5$, $\tau = 0.1$, $\kappa = 1$ and $d_A = d_B = 0.1$. Income for all countries is set to 10. Variation and probability are based on 30 samples of 20,000 simulations each. Bootstrap standard errors all $<0.01\%$.
        \end{tablenotes}
    \end{threeparttable}
\end{table}

Reduced variation after tariff is introduced could be due to a shift in resources from the final sector and consumers to the input-producing sector. The final sector doesn't face disruption. Think about this more tomorrow. Maybe not: could be the imbalance effect?

\begin{table}
    \centering
    \begin{threeparttable}
        \renewcommand{\arraystretch}{1.3}
        \caption{Input export ban example}
        \label{tab:export_ban}
        \vspace{1mm} 
        \begin{tabular}{lrrr}
            \toprule
            & \multicolumn{3}{c}{Export-banning country} \\
            \cmidrule{2-4}
            & \makecell[c]{Neither} & \makecell[c]{$A$} only & \makecell[c]{$A$ and $B$} \\
            \midrule
            \textbf{Country A} \\
            Input producers & 2 & 3 & 2 \\ 
            $\mathbb{E}(V_A)$ & 20.05 & 18.82 & 17.77 \\
            $\mathbb{E}(Q_{IA})$ & 19.66 & 25.46 & 20.41 \\
            $\mathbb{E}(Q_{FA})$ & 23.25 & 28.99 & 20.41 \\
            $\mathbb{E}(X_{FA})$ & 15.80 & 15.45 & 13.87 \\
            $CV(X_{FA})$ & 17.50\% & 15.29\% & 17.13\% \\
            $\mathbb{P}(\text{Shortfall})$ & 2.00\% & 0.37\% & 4.46\% \\ 
            \midrule
            \textbf{Country B} \\
            Input producers & 2 & 2 & 2 \\
            $\mathbb{E}(V_B)$ & 20.05 & 18.43 & 17.77 \\
            $\mathbb{E}(Q_{IB})$ & 19.66 & 18.52 & 20.41 \\
            $\mathbb{E}(Q_{FB})$ & 23.25 & 15.36 & 20.41 \\
            $\mathbb{E}(X_{FB})$ & 15.80 & 14.67 & 13.87 \\
            $CV(X_{FB})$ & 17.50\% & 15.55\% & 17.12\% \\
            $\mathbb{P}(\text{Shortfall})$ & 2.00\% & 1.60\% & 4.45\% \\ 
            \midrule
            \textbf{Country C} \\
            $\mathbb{E}(V_C)$ & 17.73 & 16.75 & 15.57 \\
            $\mathbb{E}(X_{FC})$ & 14.91 & 14.23 & 13.09 \\
            $CV(X_{FC})$ & 17.49\% & 15.30\% & 16.68\% \\
            $\mathbb{P}(\text{Shortfall})$ & 2.00\% & 0.86\% & 5.26\% \\ 
            \midrule
            \textbf{All countries} \\
            Input producers & 4 & 5 & 4 \\
            $\mathbb{E}(V)$ & 57.82 & 54.00 & 51.11 \\
            $\mathbb{E}(Q_I)$ & 39.32 & 43.97 & 40.83 \\
            $\mathbb{E}(Q_F)$ & 46.50 & 44.35 & 40.83 \\
            $CV(Q_F)$ & 17.49\% & 15.30\% & 16.68\% \\
            $\mathbb{P}(\text{Shortfall})$ & 2.00\% & 0.86\% & 5.26\% \\ 
            \bottomrule
        \end{tabular}
        \begin{tablenotes}
            \small \item Note: Calculated with $p_R = 0.5$, $\sigma = 5$, $\tau = 0.1$, $\kappa = 1$ and $d_A = d_B = 0.1$. Income for all countries is set to 10. Variation and probability are based on 30 samples of 20,000 simulations each. Bootstrap standard errors all $<0.01\%$. `Shortfall' defined as supply below half of the free trade expected level. 
        \end{tablenotes}
    \end{threeparttable}
\end{table}

If countries act solely to maximise resilience, then under certain conditions, banning exports of inputs takes on characteristics of a zero-sum game, if in the absence of an agreement to refrain from these bans. Less prisoner's dilemma and more matching pennies. The mixed strategy Nash equilibrium, based on the Table \ref{tab:export_ban} numbers for shortfall probability, is inefficient for both A and B, with the average shortfall probability exceeding that in the case with no bans. If resilience is not an objective, and only welfare matters, export bans will not eventuate.

Policies to examine:
\begin{itemize}
    \item Entry subsidy $z$: traded input (i.e. reduce the entry cost $\kappa$ and subtract the difference from income) --- do this first to illustrate the market failure (i.e. there should be an optimal non-negative entry subsidy) 
    \item Production subsidy $s$: final product, traded input, financed by tax 
    \item Export bans on traded input --- save this for last; otherwise assume at least one firm is able to export 
    \item Each of the above in cooperative vs. non-cooperative settings, i.e. is there a prisoner's dilemma? 
\end{itemize}

Criterions for resilience:
\begin{itemize}
    \item Utility
    \item Expected supply
    \item Variance of supply
    \item Likelihood of supply falling below a critical value (e.g. below half of expected supply)
\end{itemize}

Conditions to examine (maybe if time):
\begin{itemize}
    \item Country-varying disruption risk (i.e. home or foreign is higher) and productivities
    \item Differing levels of trade cost and elasticities of substitution (quick robustness check)
\end{itemize}

\section{Policy discussion}

Need better understanding of problem. Introduce green, amber, red light system for WTO. Use an organisation like IEA or OECD to identify vulnerabilities independently.

Case studies show that industrial policies can play a role in creating a new source of supply in a concentrated market. They can also lead to greater concentration. This greater concentration often lasts for a few years then dissipates. Often, the innovation that results from successful industrial policy diffuses across borders and production becomes diversified (though this requires the right trade and investment policy settings). Markets are rarely highly concentrated for extended periods as new entrants emerge. 

Some potential policy implications from the literature and case studies: 

Industrial policy will be most helpful for supply chain resilience if it does not prohibit new entrants and supports competition.  

Policies should be optimised to address a particular stated objective and consider the full set of possible instruments. E.g. for a critical good, would governments purchasing from a balanced portfolio of foreign suppliers be more cost-effective than building local production capacity? 

Target essential and vulnerable sectors, if supply chain resilience is the aim.  

For resilience, favour policy instruments that encourage firm entry (recognising there may be a trade-off between resilience and productivity). [Discuss/list some of these instruments.] 

Build in measurement and evaluation.  

Some potential policy suggestions internationally:  

Work towards better data collection (building e.g. on OECD and WTO work) on industrial policies and modelling of their international spillovers. 

Similarly, task an organisation with existing capacity in the area (e.g. OECD, IEA, WTO; possibly IPEF in future) with independently reporting on essential supply chains and identifying potential vulnerabilities.  

Following the Agreement on Agriculture and as suggested by \textcite{bown_wtoing_2019}, promote a ‘green/amber/red light’ approach in the WTO. Policies introducing a source of supply of an extremely concentrated product and using acceptable instruments may be ‘green light’. 

Develop and reinforce guardrails against policies that remove sources of supply, e.g. export restrictions outside critical shortages.

\section{Conclusion}

Summarise conclusions and suggest directions for future research. Generalise the model: expand it to many stages and many countries. More studies like the shipbuilding one; more theoretical papers like Bimpikis et al; more CGE modelling with supply disruptions like Bonadio et al. More discussion among big powers on guardrails.

\printbibliography

\end{document}